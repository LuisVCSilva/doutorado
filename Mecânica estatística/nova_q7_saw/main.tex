\documentclass[a4paper,12pt]{article}
\usepackage{amsmath, amssymb, graphicx, listings}

\title{caminhada autoevitante - variação da questão 7}
\author{Luis Vinicius}
\date{\today}

\begin{document}

\maketitle

\section{Questão 7 para caminhada autoevitante}

\subsection{Pergunta}

Considere uma caminhada autoevitante em duas dimensões: A cada passo, escolhe-se aleatoriamente uma direção permitida e dá um passo de comprimento \( l \) nessa direção, sem cruzar um caminho já percorrido.

\begin{itemize}
    \item[(a)] Escrever um código computacional para simular este passeio autoevitante.
    \item[(b)] Estimar as distribuições de probabilidade de \( r \), a distância da partícula à origem, e \( \theta \), o ângulo que o vetor posição da partícula faz com a direção horizontal.
    \item[(c)] Calcular o coeficiente de difusão desta partícula e compará-lo com o caso da caminhada aleatória simples.
\end{itemize}

\subsection{Resposta}

\subsubsection{(a) Código Computacional para Simular um Passeio Autoevitante em Duas Dimensões}

Não pode-se visitar um local já percorrido.
\subsubsection{(b) Estimando as Distribuições de Probabilidade de \( r \) e \( \theta \)}

A distância radial \( r \) da partícula à origem e o ângulo \( \theta \) do vetor posição são dados por:

\[
r = \sqrt{x^2 + y^2}
\]

\[
\theta = \tan^{-1} \left(\frac{y}{x}\right)
\]

Na questão 7 original, \( r \) cresce como \( \sqrt{t} \), aqui

\[
\langle r^2 \rangle \sim A t^{2 \alpha}
\]

Quanto vale alfa? Qual caminho cresce mais rápido?

\subsubsection{(c) Cálculo do Coeficiente de Difusão}

O coeficiente de difusão é definido como:

\[
D = \lim_{t \to \infty} \frac{\langle r^2 \rangle}{2d t}
\]

onde \( d = 2 \) para um sistema bidimensional.

Para um passeio aleatório simples, \( D \) é constante, pois \( \langle r^2 \rangle \sim t \). Nesse novo caso, o que acontece? Difunde igual, mais ou menos?
\end{document}

